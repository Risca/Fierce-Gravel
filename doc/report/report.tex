\documentclass{report}
\usepackage[utf8]{inputenc}
\usepackage[swedish,english]{babel}
\usepackage{tikz}
\usepackage{amsmath}
\usepackage{subfiles}
\usepackage{calc}
\usepackage{graphicx}
\usepackage{listings}
\usepackage{url}
\DeclareGraphicsExtensions{.pdf,.png,.jpg}
\graphicspath{{images/}}
\usetikzlibrary{calc, shapes, arrows, chains, positioning, matrix, scopes, shadows}

% Define some styles to use with tikz
\tikzstyle{cloud}     = [ellipse, draw, fill=blue!30, node distance=40pt,minimum height=20pt]
\tikzstyle{state}     = [ellipse, draw, fill=violet!30, node distance=40pt,minimum height=20pt]
\tikzstyle{subbyte}   = [rectangle, draw, text width=100pt, text centered, rounded corners=10pt, minimum height=20pt, fill=red!30]
\tikzstyle{shiftrow}  = [rectangle, draw, text width=100pt, text centered, rounded corners=10pt, minimum height=20pt, fill=orange!30]
\tikzstyle{mixcolumn} = [rectangle, draw, text width=100pt, text centered, rounded corners=10pt, minimum height=20pt, fill=yellow!30]
\tikzstyle{roundkey}  = [rectangle, draw, text width=100pt, text centered, rounded corners=10pt, minimum height=20pt, fill=green!30]
\tikzstyle{line} = [draw, -latex']

\lstset{ %
	language=VHDL,                 % choose the language of the code
	basicstyle=\footnotesize,      % the size of the fonts that are used for the code
	numbers=left,                  % where to put the line-numbers
	numberstyle=\footnotesize,     % the size of the fonts that are used for the line-numbers
	stepnumber=2,                  % the step between two line-numbers. If it is 1 each line will be numbered
	numbersep=5pt,                 % how far the line-numbers are from the code
	backgroundcolor=\color{white}, % choose the background color. You must add \usepackage{color}
	showspaces=false,              % show spaces adding particular underscores
	showstringspaces=false,        % underline spaces within strings
	showtabs=false,                % show tabs within strings adding particular underscores
	numberbychapter=false,         % Make the listing numbering to start at 1
	frame=single,                  % adds a frame around the code
	tabsize=2,                     % sets default tabsize to 2 spaces
	captionpos=t,                  % sets the caption-position to top
	breaklines=true,               % sets automatic line breaking
	breakatwhitespace=false        % sets if automatic breaks should only happen at whitespace
}

\title{AES Encryption in FPGA hardware}
\foreignlanguage{swedish}{
\author{
        Patrik Dahlström \\
        Electronic Design\\
            \and
        Daniel Josefsson\\
        Electronic Design\\
            \and
        Staffan Sjöqvist\\
        Electronic Design
}}

\date{\today}

\begin{document}
\maketitle

\subfile{abstract.tex}

\chapter{Introduction}
AES encryption is a widely recognized standard encryption that is well used in many modern applications of today. It can be found not only in modern wireless network, but also in encryption devices designed for secure storage as well as in secure wired networks.

It is this group's intention to learn how this encryption works and also how to implement it in FPGA hardware.

\chapter{Definitions}
Before describing the AES encryption, some key definitions need to be mentioned

\subfile{definitions.tex}

\chapter{Theory}
\label{sec:theory}
The AES encryption process can be divided in two different sections:
\begin{itemize}
\item Encryption process
\item Key schedule
\end{itemize}
In encryption process, an unencrypted data packet is encrypted using a cipher key and in key schedule the different round keys needed for the encryption process are generated.

\subfile{encryption_process.tex}

\subfile{key_schedule.tex}

\subfile{implementation_details.tex}

\chapter{Future improvements}
This implementation of the AES encryption is not without flaws. There are some parts that can be improved by further work. These parts include:
\begin{itemize}
\item The ability to send data, and then receive the encrypted/decrypted data, through the serial port.\footnote{It should be fairly easy to implement serial communication. Components for sending and receiving data is available, and tested, in the online sources. However, they are not used for the encryption/decryption processes at the time of writing.}
\item A more pipelined algorithm to increase throughput.
\item A more controlled encryption/decryption process. Signals should be available to indicate when the key expansion and the encryption/decryption process are complete.
\item An even more secure pipelined algorithm by implementing e.g. cipher-block chaining.
\item The current implementation can be further optimized for size and speed.
\end{itemize}

\chapter{Conclusion}
The Advanced Encryption Standard is a widely used and trusted encryption standard with no known major weaknesses.\cite{attacks} It is fast, simple and very secure. It is used, among others, by the U.S. government to secure TOP SECRET information and can be considered the most widespread encryption standard used today.

AES describes an iterating encryption/decryption process that is very simple and at the same time very fast. This makes it easy to implement and optimize the AES.

The implementation described within this document and online (available at \url{https://github.com/Risca/Fierce-Gravel/}) show how a possible implementation using FPGA hardware can be achieved.

%\section{Timeline}\label{timeline}
%The timeline below is graded in number of weeks from project start

%\begin{figure}[h]
%\centering
%\begin{tikzpicture}
%% Draw horizontal lines
%\foreach \x/\y in {0/3,3/6,6/7.5,7.5/9,9/10.5,10.5/12}
%    \draw (\x,0) -- (\y,0);
%
%% Draw horizontal lines (thicker, transparent)
%\foreach \x/\y in {0/3,3/6,6/7.5,7.5/9,9/10.5}
%    \draw[line width=2mm,gray,line cap=round,opacity=0.7] (\x+0.1,0) -- (\y-0.1,0);
%
%% Draw vertical lines
%\foreach \x in {0,3,6,7.5,9,10.5}
%   \draw[thick] (\x cm,3pt) -- (\x cm,-3pt);
%
%% Put in weeks
%\foreach \x/\xweek in {0/0,3/2,6/4,7.5/5,9/6,10.5/7}
%   \draw (\x,0) node[below=3pt]{\xweek};
%   
%% Add slanted labels for each week
%\foreach \x/\text in {0/{Start},
%                      3/{Theory},
%                      6/{Implementation},
%                      7.5/{Improvement},
%                      9/{Report},
%                      10.5/{Extra}}
%    \draw (\x,0) -- +(30:3) node[midway,sloped,above]{\text};
%
%\end{tikzpicture}
%\caption{Timeline in project weeks}
%\end{figure}

\begin{thebibliography}{9}

\bibitem{fips}
    \emph{FIPS PUB 197: the official AES standard}, 2001. Available from: \url{http://csrc.nist.gov/publications/fips/fips197/fips-197.pdf} [updated 7 January 2012]

\bibitem{attacks}
    Andrey Bogdanov, Dmitry Khovratovich, and Christian Rechberger, \emph{Biclique Cryptanalysis of the Full AES}, 2011. Available from: \url{http://research.microsoft.com/en-us/projects/cryptanalysis/aesbc.pdf} [updated 7 january 2012]

\end{thebibliography}

\end{document}
