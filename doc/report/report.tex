\documentclass[12pt]{report}

\usepackage[utf8]{inputenc}
\usepackage[swedish,english]{babel}
\usepackage{tikz}
\usepackage{amsmath}
\usepackage{subfiles}
\usepackage{calc}
\usepackage{graphicx}
\usepackage{listings}
\usepackage{hyper}
\DeclareGraphicsExtensions{.pdf,.png,.jpg}
\graphicspath{{images/}}
\usetikzlibrary{calc, shapes, arrows, chains, positioning, matrix, scopes, shadows}

% Define some styles to use with tikz
\tikzstyle{cloud}     = [ellipse, draw, fill=blue!30, node distance=40pt,minimum height=20pt]
\tikzstyle{state}     = [ellipse, draw, fill=violet!30, node distance=40pt,minimum height=20pt]
\tikzstyle{subbyte}   = [rectangle, draw, text width=100pt, text centered, rounded corners=10pt, minimum height=20pt, fill=red!30]
\tikzstyle{shiftrow}  = [rectangle, draw, text width=100pt, text centered, rounded corners=10pt, minimum height=20pt, fill=orange!30]
\tikzstyle{mixcolumn} = [rectangle, draw, text width=100pt, text centered, rounded corners=10pt, minimum height=20pt, fill=yellow!30]
\tikzstyle{roundkey}  = [rectangle, draw, text width=100pt, text centered, rounded corners=10pt, minimum height=20pt, fill=green!30]
\tikzstyle{line} = [draw, -latex']

\lstset{ %
	language=VHDL,                 % choose the language of the code
	basicstyle=\footnotesize,      % the size of the fonts that are used for the code
	numbers=left,                  % where to put the line-numbers
	numberstyle=\footnotesize,     % the size of the fonts that are used for the line-numbers
	stepnumber=5,                  % the step between two line-numbers. If it is 1 each line will be numbered
	numbersep=5pt,                 % how far the line-numbers are from the code
	backgroundcolor=\color{white}, % choose the background color. You must add \usepackage{color}
	showspaces=false,              % show spaces adding particular underscores
	showstringspaces=false,        % underline spaces within strings
	showtabs=false,                % show tabs within strings adding particular underscores
	numberbychapter=false,         % Make the listing numbering to start at 1
	frame=single,                  % adds a frame around the code
	tabsize=2,                     % sets default tabsize to 2 spaces
	captionpos=t,                  % sets the caption-position to top
	breaklines=true,               % sets automatic line breaking
	breakatwhitespace=false        % sets if automatic breaks should only happen at whitespace
}

\title{AES Encryption in FPGA hardware}
\foreignlanguage{swedish}{
\author{
        Patrik Dahlström \\
        Electronic Design\\
            \and
        Daniel Josefsson\\
        Electronic Design\\
            \and
        Staffan Sjöqvist\\
        Electronic Design
}
}
\date{\today}

\begin{document}
\maketitle

\begin{abstract}
Some abstract
\end{abstract}

\pagebreak

\chapter{Introduction}
AES encryption is a widely recognized standard encryption that is well used in many modern applications of today. It can be found not only in modern wireless network, but also in encryption devices designed for secure storage as well as in secure wired networks.

It is this group's intention to learn how this encryption works and also how to implement it in FPGA hardware.

\chapter{Definitions}
Before describing the AES encryption, some key definitions need to be mentioned

\subfile{definitions.tex}

\chapter{Theory} \label{sec:theory}
The AES encryption process can be divided in two different sections:
\begin{itemize}
\item Encryption process
\item Key schedule
\end{itemize}
In encryption process, an unencrypted data packet is encrypted using a cipher key and in key schedule the different round keys needed for the encryption process are generated.

\subfile{encryption_process.tex}

\subfile{key_schedule.tex}

\chapter{Implementation Details}
Approximately each section in chapter \ref{sec:theory} is translated into a separate VHDL entity, and this chapter will follow that structure, but with three additional sections:
\begin{itemize}
\item The \emph{Resources} Package
\item Encryption Process
\item Key Schedule
\item Top Level Entities
\item Test Benches
\end{itemize}

The \emph{resources} package was created to gather definitions of types, subtypes and components that is used throughout the project. This was done to simplify and generalize all components used.

The section Top Level Entities describes how the different components described in the sections before are interconnected.

In the section Test Benches, the various test benches used to test each component are described.

\section{The resources package}
The VHDL standard defines an elegant way to summarize all different types, subtypes, functions, components, etc. in what is called a \emph{package}. This project have one package for all its parts --- The \emph{resources} package.

This package begins with constants and type definitions, as shown in listing \ref{list:constants} below.

\lstinputlisting[firstline=7,
                 firstnumber=7,
                 lastline=26,
                 label={list:constants},
                 caption={FG\_package.vhd: Constants and type definitions}]
{FG_package.vhd}


\subfile{impl_encryption_process.tex}

\subfile{impl_key_schedule.tex}

\section{Top level entities}

\section{Test benches}


\chapter{Future improvements}


\chapter{Conclusion}

%\section{Timeline}\label{timeline}
%The timeline below is graded in number of weeks from project start

%\begin{figure}[h]
%\centering
%\begin{tikzpicture}
%% Draw horizontal lines
%\foreach \x/\y in {0/3,3/6,6/7.5,7.5/9,9/10.5,10.5/12}
%    \draw (\x,0) -- (\y,0);
%
%% Draw horizontal lines (thicker, transparent)
%\foreach \x/\y in {0/3,3/6,6/7.5,7.5/9,9/10.5}
%    \draw[line width=2mm,gray,line cap=round,opacity=0.7] (\x+0.1,0) -- (\y-0.1,0);
%
%% Draw vertical lines
%\foreach \x in {0,3,6,7.5,9,10.5}
%   \draw[thick] (\x cm,3pt) -- (\x cm,-3pt);
%
%% Put in weeks
%\foreach \x/\xweek in {0/0,3/2,6/4,7.5/5,9/6,10.5/7}
%   \draw (\x,0) node[below=3pt]{\xweek};
%   
%% Add slanted labels for each week
%\foreach \x/\text in {0/{Start},
%                      3/{Theory},
%                      6/{Implementation},
%                      7.5/{Improvement},
%                      9/{Report},
%                      10.5/{Extra}}
%    \draw (\x,0) -- +(30:3) node[midway,sloped,above]{\text};
%
%\end{tikzpicture}
%\caption{Timeline in project weeks}
%\end{figure}

\end{document}
